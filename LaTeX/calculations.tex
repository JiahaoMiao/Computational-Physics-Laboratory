\documentclass[main.tex]{subfiles}

\begin{document}

\section{Benchmarks and Calculations}

Armed with the framework for constructing Feynman diagrams and applying Feynman rules, we can now proceed to compute the 
scattering amplitudes for gluon interactions in (QCD) and verify their properties. 

Before diving into the calculations, it's important to understand the expected complexities arising from the generated amplitudes.

\subsection{Number of Feynman Diagrams and Color Structures}

The total number of tree-level Feynman diagrams for $n$ gluons with 3 and 4 point vertices can be computed using a recurrence relation\cite{OEIS:A268163} and shown in \Cref{lst:tree-diagrams}.
Among these, the number of diagrams featuring only 3-point vertices is given by $(2n-5)!!$ (double factorial) \cite{OEIS:A001147}.Each of these diagrams corresponds to a unique color substructure. 
The remaining diagrams, which include at least one 4-point vertex, are generated by adding a 4-point vertex to existing diagrams with only 3-point vertices and 
they do not introduce any new color substructures.

In general, for $n$ gluons, there are $(2n-5)!!$ color substructures.These takes the form of contractions of $n-2$ structure constants $f^{abc}$ with $n-3$ dummy(summed over) indices $a_i$ and $n$ distinct
indices $c_i$, e.g $f^{c_2 c_1 a_1}f^{c_4 c_3 a_1}$ for $n=4$, $f^{c_1 a_2 a_1} f^{c_3 c_3 a_1} f^{c_5 c_4 a_2}$ for $n=5$, and so on. 

\begin{table}[htbp]\centering
\caption{Number of Tree-level Feynman Diagrams for $n$ gluons. The first column indicates the number of external gluons,
the second column shows the total number of diagrams, the third column counts the diagrams with only 3-point vertices, and the fourth column counts those with at least one 4-point vertex.}
\label{tab:tree-diagrams}

    \begin{tabular}{|c|c|c|c|}
        \hline
        \textbf{n} & \textbf{Number of Diagrams} & \textbf{3-point Vertices only} & \textbf{with 4-point Vertices} \\
        \hline
        4 & 4 & 3 & 1 \\
        5 & 25 & 15 & 10 \\
        6 & 220 & 105 & 115 \\
        7 & 2485 & 945 & 1540 \\
        8 & 34300 & 10395 & 23905 \\
        \hline    
    \end{tabular}
\end{table}

\subsection{Amplitude Generation}

The process of generating the scattering amplitude is straightforward. We generate the Feynman diagrams using the `generatediagrams` function (\cref{lst:generate-diagrams}), sum 
the contributions from each diagram, and then apply the Feynman rules to obtain the amplitude. 

The computational resources required for generating the diagrams only, specifically the time and memory, are summarized in Table~\ref{tab:computational-resources},
which shows that time and memory usage grows exponentially with the number of gluons involved in the scattering process.
Applying the Feynman rules further increases the computational cost, as shown in the same table.


\begin{table}[htbp]
    \centering
    \caption{Computational Resources for Feynman Diagram Generation and Amplitude. The Amplitude
    is generated by applying the operations: Total, FeynmanRules and Expand. The result for 
    8 gluons is not available (TBD = To Be Determined) due to the high computational cost.} 
    \label{tab:computational-resources}
    \begin{tabular}{|c|c|c|c|c|c|}
        \hline
        \multicolumn{2}{|c|}{\textbf{Number of Gluons}} & \multicolumn{2}{|c|}{\textbf{Generated Diagrams}} & \multicolumn{2}{|c|}{\textbf{Amplitude}} \\
        \cline{1-6} % To create a horizontal line across all columns
        \textbf{n} & \textbf{Number of Diagrams} & \textbf{Time (s)} & \textbf{Memory (MB)} & \textbf{Time (s)} & \textbf{Memory (MB)} \\
        \hline
        4 & 4 & 0.0011 & 0.01 & 0.004 & 0.096 \\
        5 & 25 & 0.0100 & 0.12 & 0.051 & 2.87 \\
        6 & 220 & 0.1153 & 1.51 & 6.503 & 104.52 \\
        7 & 2485 & 1.5998 & 22.99 & 1112.738 & 4430.25 \\
        8 & 34300 & 27.5766 & 404.70 & TBD & TBD \\
        \hline
    \end{tabular}
\end{table}

After generating the amplitude, there are several important properties that we can verify to ensure the correctness of our calculations.


\subsection{Symmetry under Exchange of External Legs}
The scattering amplitude for gluon interactions should be symmetric under the exchange of external legs.

To veriry this property, the following function are defined:

\begin{itemize}
    \item \textbf{swapTwoParticles[amp_,i_,j_]}: This function takes an amplitude and swaps the $i$-th and $j$-th external legs, 
    by replacing the corresponding Lorentz, color and momentum labels in the amplitude expression.
    \item \textbf{pairmap[list1_, list2_]} : This function takes two lists and returns a list of pairs, where each pair consists of an element from the first list and the corresponding element from the second list.
\end{itemize}

In the two to two scattering case ($n=4$), there are only 3 color substructures and after exchangin external legs, these color substructures
are permuted among themselves, their Lorentz coefficients also changes accordingly, but the overall amplitude remains unchanged.

For $n \ge 5$, the permutation are more complex and the function \textbf{pairmap} is needed to keep track of which term goes where. Then verify that the amplitude remains unchanged 
by subtracting the original color substructure Lorentz coefficients from the swapped ones and checking if the result is zero.

From this we can conclude that the amplitude is symmetric under the exchange of external legs and that
all the color substructures are permuted among themselves, so only a single color substructure is needed and rest can be generated
by substitution rules, which is significatly more efficient than all the diagrams, substituting Feynman rules, summing over all diagrams 
and then collecting the color substructures. 

It is possible to generate the Lorentz coefficient for a single color substructure by generating only the diagrams
contributing to that color substructures. This has been implemented in the function \texttt{generatemasterdiagram[n_]}, which
generates diagrams which contributes to the color substructure $f^{c_1 c_2 a_1}f^{c_3 a_1 a_2}\cdots f^{c_n c_{n-1} a_1}$.

The bottleneck given by the generation of Feynman diagrams has thus been reduced to a symmetry problem, given this color substructure one has 
to find the correct permutation to obtain all the other color substructures. 

\subsection{Ward Identities}
The Ward identities are a set of relations that must be satisfied by the scattering amplitudes in gauge theories.
In the case of gluon scattering, the Ward identity states that the amplitude must vanish when any external on-shell gluon polarization is substituted
with its momentum.

While the overall scattering amplitude for gluon interactions satisfies the Ward identity, the presence of distinct color substructures 
within the amplitude poses a unique challenge. Since these color substructures prevent a direct summation of the various 
Lorentz structures, each individual color substructure must inherently be gauge invariant for the full amplitude to satisfy the identity.

However, the $(2n-5)!!$ color substructures are not all linearly independent, as they are related by the Jacobi identity as shown in Equation \cref{eq:jacobi-identity} for the structure constants $f^{abc}$, so 
they are not gauge invariant. These non gauge invariant dependent substructures still vanish when all external polarization vectors are simultaneously substituted with their respective momentum vectors.

\begin{equation} \label{eq:jacobi-identity}
    f^{aeb}f^{ecd} + f^{ade}f^{ecb} - f^{ace}f^{edb} = 0
\end{equation}

For $n$ gluons, each of these color substructures has $n-3$ dummy indices, so each term can generate $n-3$ jacobi identities, though these may not all be unique.
These color substructures can be mapped to variables using the Mathematica function \textbf{MapIndexed} to variables $v[i]$ so that Mathematica can work with them.
The Jacobi identities can then be transformed into a system of equations and solved using \textbf{Solve}.

In conclusion, the initial set of $(2n-5)!!$ color substructures reduces to $(n-2)!$ independent color substructures \cite{DELDUCA200051}.
These independent color substructures are each accompanied by their respective Lorentz structures, which are inherently gauge invariant and collectively satisfy the Ward identity.

Thanks to the symmetry under exchange, it is sufficient to verify the Ward identity for just one of these independent color substructures.

Necessary conditions for the Ward identity are momentum conservation and the transversality of the polarization vectors.

All the steps above are implemented in the module \texttt{VerifyWard[ngluons]}, in \Cref{tab:ward-identity-verification} we summarize the performance of the verification of the Ward identity for $n=4,5,6$ gluons.
Notice that times and memory usage scales exponentially with the number of gluons and simplification times depends on the size of the expressions,

\begin{table}[htbp]
    \centering
    \caption{Performance Metrics for Ward Identity Verification in Gluon Scattering}
    \label{tab:ward-identity-verification}
    \begin{tabular}{|c|c|c|c|}
        \hline
        \textbf{Contracted  with} & \textbf{Contraction Time (s)} & \textbf{Memory (MB)} & \textbf{Simplification Time (s)} \\
        \hline
        \multicolumn{4}{|c|}{\textbf{$n=4$ Gluons}} \\
        \hline
        4p 0$\epsilon$ & 0.0023 & 0.00002 & 0.00008 \\
        3p 1$\epsilon$ & 0.0038 & 0.00350 & 0.00008 \\
        2p 2$\epsilon$ & 0.0069 & 0.02103 & 0.00016 \\
        1p 3$\epsilon$ & 0.0099 & 0.04923 & 0.00020 \\
        \hline
        \multicolumn{4}{|c|}{\textbf{$n=5$ Gluons}} \\
        \hline
        5p 0$\epsilon$ & 0.093 & 0.077 & 0.00034 \\ 
        4p 1$\epsilon$ & 0.134 & 0.162 & 0.00062 \\
        3p 2$\epsilon$ & 0.203 & 0.515 & 0.00160 \\
        2p 3$\epsilon$ & 0.281 & 0.952 & 0.00327 \\
        1p 4$\epsilon$ & 0.376 & 1.561 & 0.07672 \\
        \hline
        \multicolumn{4}{|c|}{\textbf{$n=6$ Gluons}} \\
        \hline
        6p 0$\epsilon$ & 4.04 & 2.67 & 0.23 \\
        5p 1$\epsilon$ & 5.08 & 4.52 & 0.79 \\
        4p 2$\epsilon$ & 7.03 & 11.48 & 9.23 \\
        3p 3$\epsilon$ & 9.64 & 22.11 & 30.04 \\
        2p 4$\epsilon$ & 12.59 & 37.03 & 151.34 \\
        1p 5$\epsilon$ & 15.70 & 54.53 & 183.86 \\
        \hline
    \end{tabular}
    \caption*{* The "Memory" column indicates the memory (in Megabytes) required to store the expression just before the simplification step. 
    The "Contraction Time" includes applying momentum conservation, the transverse condition, and contractions. The "Simplification Time" reflects the duration of the \texttt{Simplify} operation.}
\end{table}

\subsection{Modulus Squared of the Amplitude}

Calculating the modulus squared of the amplitude is typically the most computationally intensive step in scattering amplitude computations. 
For gluon scattering, various strategies can be employed to manage this complexity, primarily by organizing the amplitude and its color 
substructures differently.

\begin{enumerate}
    \item \textbf{Direct Squaring of Feynman Diagrams:} This is the most straightforward method, involving the direct squaring of the sum of all Feynman diagrams.

    \item \textbf{Collecting Non-Independent Color Substructures:} A more efficient approach involves collecting all $(2n-5)!!$ non-independent color substructures before squaring their sum. 
    This strategy significantly improves efficiency by allowing for early cancellations, thereby substantially reducing the size of intermediate expressions and overall memory footprint.

    \item \textbf{Utilizing Jacobi Identities for Independent Color Substructures:} This method further reduces the number of color substructures to $(n-2)!$ by employing Jacobi identities to derive a basis of independent color structures. 
    While this approach minimizes the number of terms that need to be computed, it can lead to coefficients with larger individual expressions. This presents a trade-off between the number of calculations performed and the symbolic size of the resulting terms.
\end{enumerate}

To illustrate the computational characteristics of these approaches, Table \ref{tab:modulus-squared-performance} presents performance metrics,
including memory usage and computation time, for $n=4$ and $n=5$ gluon scattering processes.
The memory usage is detailed for various stages of the calculation: the size of a single selected term, its size after contraction with polarization sums (using specific choices for polarization vectors $n^{\mu}$ aligned with external momenta for simplification),
the size after expanding against a conjugate term, the size after subsequent contraction, after renaming Lorentz invariants, and finally, after full simplification. Lastly,
the total time taken to compute the modulus squared is also recorded.

\begin{table}[htbp]
    \centering
    \caption{Performance Metrics for Modulus Squared Calculation in Gluon Scattering}
    \label{tab:modulus-squared-performance}
    \begin{tabular}{|c|c|c|c|}
        \hline
        \textbf{Operation Stage} & \textbf{Feynman Diagrams} & \textbf{Non-Independent Basis} & \textbf{Independent Basis} \\
        \hline
        \multicolumn{4}{|c|}{\textbf{$n=4$ Gluons}} \\
        \hline
        Total Terms & 10 & 6 & 3 \\
        Single Term (MB) & 0.028 & 0.025 & 0.050 \\
        Times Polarization (MB) & 0.945 & 0.693 & 1.512 \\
        Times Conjugate (MB) & 15.449 & 4.186 & 32.592 \\
        Contract (MB) & 1.364 & 0.358 & 2.220 \\
        Mandelstam (MB) & 0.389 & 0.116 & 0.622\\
        Simplify (MB) & 0.005 & 0.003 & 0.006\\
        All Contributions (MB) & 0.026 & 0.023 & 0.018 \\
        Time Taken (s) & 3.318 & 1.106 & 3.184 \\
        \hline
        \multicolumn{4}{|c|}{\textbf{$n=5$ Gluons}} \\
        \hline
        Total Terms & 325 & 120 & 21 \\
        Single Term (MB) & 0.134 & 0.133 & 0.633 \\
        Times Polarization (MB) & 10.455 & 8.970 & 56.295 \\
        Times Conjugate (MB) & 4893.640 & 3695.500 & TBD \\
        Contract (MB) & 6.325 & 1.553 & TBD \\
        Mandelstam (MB) & 7.760 & 1.681 & TBD \\
        Simplify (MB) & 0.072 & 0.059 & TBD \\
        All Contributions (MB) & TBD & 6.867 & TBD \\
        Time Taken (min) & TBD & $\sim$50 & TBD \\
        \hline
    \end{tabular}
    \caption*{Note: the most memory-intensive operation is the expansion against a conjugate term, which is necessary for Contraction to act on the expression.}
\end{table}

The correctness of the modulus squared calculation has been verified by comparing with the results in \cite{Ellis:1996mzs,BERENDS1981124}.



\end{document}