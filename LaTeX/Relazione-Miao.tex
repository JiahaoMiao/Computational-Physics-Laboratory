\documentclass[10pt,english]{article}

\usepackage{style/packages}
\usepackage{style/relazione}

\addbibresource{bibliography.bib} % The file containing the bibliography

\newcommand{\myabstract}{
    \begin{abstract}
    \normalsize This report explores the use of the Mathematica software package to compute tree-level gluon scattering amplitudes in perturbative quantum chromodynamics (QCD),
    leveraging its powerful algebraic capabilities.
    The main focus is on utilizing Mathematica's list manipulation functions to represent Feynman diagrams, applying Feynman rules to evaluate the corresponding scattering amplitudes. 
    Emphasis is placed on the computational techniques used to automate and simplify these calculations and resources consumed.
    Key properties such as Ward identities and symmetry under the exchange of external legs are verified to ensure the robustness of the calculations. 
    While Mathematica proves efficient in handling complex algebraic expressions and streamlining the evaluation of QFT amplitudes, 
    the report also highlights the exponential growth in computational resource requirements as the number of external legs increases.  
    \end{abstract}
}    

\begin{document}

\maketitle

\subfile{introduction}

\subfile{implementation}

\subfile{calculations}

\section{Conclusions}

This report highlights Mathematica's capability in efficiently computing tree-level gluon scattering amplitudes in QCD. 
We developed a framework that automates the generation of Feynman diagrams and the application of Feynman rules, 
significantly streamlining the derivation of amplitudes. However, the growth of the number of diagrams is super-exponential, demonstrating 
that while computational tools are essential, they alone cannot manage the rapidly increasing algebraic complexity.

We validated the framework's robustness by confirming its symmetry under external leg exchange and its consistency with Ward identities, 
both of which exhibited exponential resource scaling. Ultimately, while Mathematica proves to be a powerful tool for addressing this problem, 
the exponential complexity inherent in gluon scattering amplitudes necessitates a more sophisticated approach that leverages symmetry to manage 
the growth effectively

% bibliography 
\printbibliography

\end{document}

