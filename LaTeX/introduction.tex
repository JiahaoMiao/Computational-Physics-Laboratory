\documentclass[main.tex]{subfiles}
\begin{document}

\section{Introduction}
\pagenumbering{arabic}
\setcounter{page}{1}

Quantum Chromodynamics (QCD), the fundamental theory of the strong interaction, describes the behavior of quarks and gluons. 
Understanding high-energy collider phenomena, such as jet production, necessitates the precise calculation of scattering amplitudes 
involving these fundamental particles. 

While the theoretical framework for computing these amplitudes using Feynman diagrams is well-established, 
the practical execution of these calculations quickly becomes an arduous and error-prone task due to the rapidly increasing number of diagrams, 
the complexity of algebraic expressions, and the intricate handling of momentum, color, and Lorentz indices.

For gluon scattering in QCD, the number of tree-level Feynman diagrams grows more than factorially with the number of external legs $n+1$, 
following an asymptotic growth pattern given by (see notebook):

\begin{equation}
    \order{\left(\frac{9 \sqrt{3}+12}{11} \right)^n \frac{n!}{n^{3/2}}}   
\end{equation}

For instance, the number of tree-level diagrams for the first few $n+1$ gluons can be computed using a recurrence relation (\Cref{lst:tree-diagrams}) 
or counted directly after generating the diagrams in Mathematica.

\begin{lstlisting}[language=Mathematica,caption = {Number of tree-level diagrams for $n+1$ gluons}, label = {lst:tree-diagrams}]
    a[0] = 0; a[1] = 1; a[2] = 1;
    a[n_] := a[n_]:=a[n]=(12(2n-3)a[n-1]+(3n-5)(3n-7)a[n-2])/11;
    Table[a[n], {n, 3, 12}];
    Output: {4, 25, 220, 2485, 34300, 559405, 10525900, 224449225, 5348843500, 140880765025}
\end{lstlisting}

This increasing complexity, particularly in perturbative QCD, makes a computational approach essential.
For this project, Mathematica \cite{Mathematica} was selected as the primary computational environment due to its unparalleled symbolic manipulation capabilities, which are uniquely well-suited 
for the challenges of calculating Feynman amplitudes. Unlike compiled languages such as C++ or Fortran, which excel in numerical computations and low-level 
control, Mathematica provides a high-level, interactive environment that natively understands and operates on symbolic expressions.

% Consider the task of applying Feynman rules: a C++ implementation would require the manual definition and manipulation of complex algebraic structures, 
% potentially involving custom-built tensor classes, symbolic differentiation, and intricate expression simplification algorithms. This demands significant 
% upfront development time for a robust symbolic engine, an endeavor that often constitutes a research project in itself. In contrast, Mathematica offers 
% these functionalities out-of-the-box. Its ability to perform automatic algebraic simplification, handle multi-indexed tensors, solve systems of symbolic 
% equations (e.g., for momentum conservation), and perform symbolic differentiation and integration provides a powerful toolkit that directly maps to the 
% requirements of amplitude calculation. This allows us to focus more on implementing the physics rules and explore symmetry relations rather than building
% the underlying symbolic infrastructure from scratch.

This report details the computational implementation of gluon amplitude calculations in QCD using Mathematica. 
We will demonstrate how Mathematica's unique strengths in symbolic computation, automated algebraic simplification, and high-level programming facilitate 
the systematic construction of Feynman diagrams, the automated assignment of momenta, the application of Feynman rules, and ultimately, the derivation of 
explicit amplitude expressions and modulus squared.

The subsequent sections will walk through the design of our data structures, the algorithms for momentum assignment and 
constraint solving, the implementation of Feynman rules as substitution rules, color algebra handling, verification
of key properties such as Ward identities, and computation of the modulus squared.

All the code used in this report were run in a laptop with AMD Ryzen 7 5800H CPU and 16 GB RAM.


\end{document}